Kurzanleitung:

% ist das Kommentarzeichen in Latex. Alles was dahinter steht wird ignoriert. Man kann also seinen eigenen Text kommentieren 
% ohne das diese Kommentar später im fertigen Dokument erscheinen

Mit dem folgenden Befehl wird das eigentliche PDF-Dokument erzeugt.
./make_pdf.sh

Das Ergebnis wird inklusive aller erzeugten Dateien in den Ordner output abgelegt. 
Man kann alles darin löschen (sofern man nicht selbst etwas hineingelegt hat ;-)
Falls man mal alles gelöscht hat, dann müssen zwei Durchläufe gemacht werden, damit das Inhaltsverzeichnis und alle anderen Verzeichnisse erstellt werden können.
Der erste Durchlauf legt die .toc Datei an, die im zweiten Durchlauf für die Erstellung der Verweise (u.a. der Verzeichnisse) genutzt wird.
Verändert man große Teile in der Gliederung kann es ebenfalls notwendig sein, zwei Durchläufe zu starten, damit alles aktuell ist.

Das Hauptdokument ist Dokumention_main.tex
Hier werden alle anderen Dokumente mit dem Befehl \input{DATEINAME} eingebunden. Die Endung .tex kann dabei entfallen.
Im folgenden einige besondere Dateien:

preinput.tex 
Sie enthält alle benötigten Pakete (\usepackage[PAKETNAME]) bzw. neudefiniert Befehle (\newcommand). Will man 
eigene Befehle definieren, um sich die Arbeit zu vereinfachen, dann sollte man dies hier tun, damit man den Überblick nicht 
verliert.

Die anderen Dateien sind eigentlich durch den Dateinamen selbstklärend. Man kann die Arbeit auch in Unterordner verteilen.
Z.B. alle Bilder in picture legen. Dies kann man dann nutzen, um per \newcommand einen Befehl zu kreieren, der 
nur wenige Parameter benötigt. Im vorliegend Beispiel wurden die Ordner verwendet, um für jeden Teilnehmer einen eigenen Bereich
zu schaffen. Der Fantasie sind (fast) keine Grenzen gesetzt.


Infos für weitere Autoren:
Inhaltsverzeichnis wird aus allen \section und \subsection automatisch erzeugt.

Bitte folgende ``Befehle'' für das Schreiben benutzen.. Dann sieht es einheitlich aus...
\section{TEXT} -> Neuer Abschnitt (z.B. 1. Projektziel)
\subsection{TEXT} Neuer Unterpunkt (z.B. 1.1 Projektumgebung)
\par\bigskip -> Absatz
\newline -> Zeilenumbruch

Stichpunkte:
\begin{itemize} -> Start
\item Ein Stichpunkt -> Stichpunkt
\item Noch ein Stichpunkt -> Stichpunkt
\end{itemize} -> Ende

Unterstrich bitte mit \ maskieren -> also \-
Überschriften und Unterüberschriften:
\section{} -> 1 Ebene
\subsection{} -> 2 Ebene
\subsubsection{} -> 3 Ebene
\paragraph{} -> 4 Ebene
\subparagraph{} -> 5 Ebene
BITTE HINTER \paragraph{} und \subparagraph{} IMMER folgendes setzen $\;$\\.. Sonst gibt es keinen
Zeilenumbruch.. also sieht dann so aus \paragraph{}$\;$\\

Bilder einfügen an Ort und Stelle per zwang!!!!!!! Gefunden auf ftp://ftp.rrzn.uni-hannover.de/pub/mirror/tex-archive/info/l2picfaq/german/l2picfaq.pdf
\begin{center}
\begin{minipage}{0.7\linewidth}
\centering
\includegraphics{Bild}%
\captionof{figure}[erscheint im Abbildungsverzeichnis]{direkte Bildunterschrift}%
\end{minipage}
\end{center}

Zitate
\begin{quote}
 Eigentliches Zitat
\end{quote}

Fett schreiben
\textbf{Fett geschriebens Element}\\

% URL Einbinden immer so!!!!!!
\url{http://www.bla_bla_bla}
Latex hat Probleme mit der Darstellung von bestimmten Zeichen wie _ oder ähnlichem

Section NICHT zählen section*{}



