\newpage
%Abkürzungsverzeichnis
% wird in Main-Dokument eingebunden SICH HIER IM MAIN DOKUMENT
%\acro{Abkürzung}{Ausgeschrieben} -> Neue Abkürzung anlegen
% IM TEXT \ac{Abkürzung} -> Gibt EINMAL vollen Text mit Abkürzung in Klammern, danach %nur noch Abkürzung an
% Beispiel: \ac{cloop}
\section{Abkürzungsverzeichnis}
\begin{acronym}[Bash]
\acro{cloop}{compressed loop device}
\acro{cpu}{change passwort utility}
\acro{CSS}{Cascading Style Sheets}
\acro{DN}{Distinguished Name}
\acro{HTML}{Hypertext Markup Language}
\acro{KVM}{Kernel-based Virtual Machine}
\acro{LDAP}{Lightweight Directory Access Protocol}
\acro{LDIF}{LDAP data interchange format}
\acro{Linbo}{Linux-based Network Boot Operating System}
\acro{LTS}{Long Term Support}
\acro{NIS}{Network Information Service}
\acro{OS}{Operating System}
\acro{PXE}{Preboot Execution Environment}
\acro{RDN}{Relative Distinguished Name}
\acro{SheilA}{Selbstheilende Arbeitsstation}
\acro{SSH}{Secure Shell}
\acro{ul}{Unordered List}
\acro{li}{List Item}
\acro{ol}{Ordered List}
\acro{URI}{Uniform Resource Identifier}
\acro{VM}{virtuelle Maschine}
\acro{W3C}{World Wide Web Consortium}
\end{acronym}
\newpage
