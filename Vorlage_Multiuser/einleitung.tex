\section{Einleitung}
\subsection{Vorwort}
Hier kommt ein nettes Vorwort hin. 

Neue Absätze werden durch zwei 'returns' eingefügt.

\subsection{Projektbeschreibung}
Natürlich sollte das Projekt kurz beschrieben werden. Das könnte man an dieser Stelle tun. Dieser Text sollte 
aber nicht zu lang werden, sonst bleibt nichts mehr für die eigentliche Dokumentation.


\par\bigskip
Den Befehl \code{\textbackslash par\textbackslash bigskip} kann man verwenden, wenn man einen größeren Abstand zwischen den Absätzen haben
möchte. Dies sollte aber nur in Ausnahmen genutzt werden, da sonst das Schriftbild darunter leidet.

Vielleicht kommen an dieser Stelle auch schon die ersten Abkürzungen wie \ac{LDAP}.

\subsection{Aufgabenverteilung}
Die Aufgaben der einzelnen Projektmitglieder sollten klar benannt werden.
Dies könnte hier geschehen.
\begin{itemize}
\item Projektteil 1 blablabla
\item Projektteil 2 blablabla
\item Projektteil 3 blablabla
\end{itemize}

Und schon ist unsere Einleitung fertig. \Winkey
