\newpage
\section{Projektumgebung}
Im Kapitel Projektumgebung kann der Umfeld in dem das Projekt stattfand beschrieben werden. 
Hier können natürich wieder einige Unterkapitel kommen. Diese werden mit dem Befehl 
\code{\textbackslash subsection\{Unterkapiteltext\}} eingefügt.

\subsection{Unterkapitel zur Projektumgebung}
Eventuell kommt man mit einem Unterkapitel nicht aus. Kein Problem dann fügt man eben noch weitere hinzu.

\subsubsection{Und noch ein Unterkapitel zur Projektumgebung}
bla bla bla

\subsubsection{Soviele Unterkapitel zur Projektumgebung}
bla bla bla

\textbf{Extrabereich1}$\;$\\
Wenn man mal etwas fett aber ohne eigene Überschrift verwenden möchte.

\newpage
\textbf{Extrabereich2}$\;$\\
Und noch etwas vervorgehobenes
Das folgende Bild wurde durch ein neues Kommando eingefügt:
\code{\textbackslash picinsert\{0.5\}\{hyperbel\_kreis\_ellipse\}\{Hyperbel mit Kreis\}}
Dabei werden die Parameter in \{\}

\picinsert{0.5}{hyperbel_kreis_ellipse}{Hyperbel mit Kreis}





\subsubsection{Aufzählungen und wie man sie anlegt}
Eine tolle Sache sind auch Aufzählungen. Diese können per \code{\textbackslash begin\{itemize\}} eingeleitet.
Die einzelnen Punkte werden dann mit \code{\textbackslash item} Zeile für Zeile eingefügt.
Wo ein \code{begin} ist gibt es natürlich auch ein \code{end}.
\begin{itemize}
\item Punkt 1
\item Punkt 2 
\item Punkt 3
\end{itemize}
