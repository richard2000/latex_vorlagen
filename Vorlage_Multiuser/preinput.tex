\documentclass[listof=numbered,12pt,a4paper]{scrartcl}
\usepackage[scaled]{helvet}
\usepackage[T1]{fontenc}
\usepackage[utf8x]{inputenc}
\usepackage[ngerman]{babel}
\usepackage{geometry}
\usepackage{setspace}
\usepackage[printonlyused, withpage]{acronym}
\usepackage{graphicx}
\usepackage{graphics}
\usepackage{capt-of}
\usepackage[absolute]{textpos}
\usepackage[automark]{scrpage2}
\usepackage[procnames]{listings}
\usepackage{color}
\usepackage{url}
\usepackage{pdfpages}
\usepackage{hyperref}
% zwei Bilder nebeneinander darstellen
\usepackage{subfigure}
%\usepackage{titlesec}
%smiley-Packet
\usepackage{tikzsymbols}
\setkomafont{sectioning}{\bfseries} 
\setcounter{tocdepth}{5}
\setcounter{secnumdepth}{5}
\geometry{a4paper,left=30mm,right=20mm}
% beheben ds tightlist-Errors
\def\tightlist{}



% Nach \title{}
\DeclareFixedFont{\ttb}{T1}{txtt}{bx}{n}{12} % for bold
\DeclareFixedFont{\ttm}{T1}{txtt}{m}{n}{12}  % for normal
% Custom colors
\definecolor{deepblue}{rgb}{0,0,0.5}
\definecolor{deepred}{rgb}{0.6,0,0}
\definecolor{deepgreen}{rgb}{0,0.5,0}

\usepackage{listings}

% Python style for highlighting
\newcommand\pythonstyle{\lstset{
language=Python,
basicstyle=\ttm,
otherkeywords={self},             % Add keywords here
keywordstyle=\ttb\color{deepblue},
emph={MyClass,__init__},          % Custom highlighting
emphstyle=\ttb\color{deepred},    % Custom highlighting style
stringstyle=\color{deepgreen},
frame=tb,                         % Any extra options here
showstringspaces=false            %   
}}


% Python environment
\lstnewenvironment{python}[1][]
{
\pythonstyle
\lstset{#1}
}
{}

% Python for external files
\newcommand\pythonexternal[2][]{{
\pythonstyle
\lstinputlisting[#1]{#2}}}

% Python for inline
\newcommand\pythoninline[1]{{\pythonstyle\lstinline!#1!}}

% \code{TEXT} um TEXT als monospace (courier new) dazustellen
\newcommand{\code}[1]{\texttt{#1}}


\newcommand{\twopicinsert}[7] {
%\twopicinsert{weite1}{dateiname1 (ohne png)}{Abbildungbeschriftung1}{weite2}{dateiname2}{Abbildungbeschriftung2}{Gesamtbeschrift}}}
  \begin{figure}[h]
  \subfigure[#3]{\includegraphics[width=#1\textwidth]{./res/#2.png}}
  \subfigure[#6]{\includegraphics[width=#4\textwidth]{./res/#5.png}} 
  \caption{#7\label{fig:#3}} 
  \end{figure}
}


\newcommand{\picinsert}[3] {
% \picinsert{0.5}{dateiname (ohne png)}{Abbildungbeschriftung}
  \begin{figure}[h]
  \centering
  \includegraphics[width=#1\textwidth]{./res/#2.png}
% Bildtitel zur Referenzierung
  \caption{#3\label{fig:#2}}
  \end{figure}
}


\usepackage{graphicx}
% Redefine \includegraphics so that, unless explicit options are
% given, the image width will not exceed the width of the page.
% Images get their normal width if they fit onto the page, but
% are scaled down if they would overflow the margins.
\makeatletter
\def\ScaleIfNeeded{%
  \ifdim\Gin@nat@width>\linewidth
    \linewidth
  \else
    \Gin@nat@width
  \fi
}
\makeatother
\let\Oldincludegraphics\includegraphics
{%
 \catcode`\@=11\relax%
 \gdef\includegraphics{\@ifnextchar[{\Oldincludegraphics}{\Oldincludegraphics[width=\ScaleIfNeeded]}}%
}




