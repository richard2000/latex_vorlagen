\newpage
\section{Realisierung}
\subsection{Mal fremde ``Federn'' einfügen}
So könnt man einen fremden Text zitieren:
\begin{quote}
``Hier steht der fremde Text der zitiert werden soll. bla bla bla ''
\newline
\url{https://de.wikipedia.org/wiki/Projekt} 20.05.2014 18:48 Uhr
\end{quote}
\subsubsection{Und noch ein Kapitel}
\paragraph{Wir kommen mit den Ebenen nicht aus}$\;$\\
bla bla bla 
\paragraph{Relative Distinguished Name}$\;$\\
Und hier noch ein Literaturverweis auf [Scheibner2011]
bla bla bla 

Wenn man Sourcecode schreibt soll dieser natürlich gut aussehen in der Dokumentation. Dies kann man z.B. so 
darstellen wie hier:
\begin{python}
dn: dc=schule, ou=People,uid=ti2a
objectClass: account
uidNumber: 2000
description: 2015
\end{python}

Die Darstellung hilft zur Strukturierung des Textes und gibt Übersicht.
\begin{python}
import ldap
\end{python}

\begin{python}
return template('static/pwneusetzen.tpl', user = user, 
			fehler = '', success = 'true',
			liste = liste1)
else:
	return template('static/pwneusetzen.tpl', 
	user = user, fehler = 'true', 
	success = '', liste = liste1)
\end{python}

Manchmal möchte man aber nicht einen ganzen Abschnitt als Code darstellen, sondern nur ein paar Befehle 
innerhalb des Textes \code{\textbackslash pythoninline} hilft dabei einen Befehl wie \pythoninline{command} 
entsprechend darzustellen.

Quellcode im Python-Skript:
\begin{python}
if inputcheck(uid, password) == True:
	if syntaxchecken(uid) == True:
		if userexsits(uid) == False:
\end{python}


